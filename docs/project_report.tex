\documentclass[11pt,a4paper]{report}

% ============================================================
% PACKAGES
% ============================================================
\usepackage[utf8]{inputenc}
\usepackage[T1]{fontenc}
\usepackage[french]{babel}
\usepackage{graphicx}
\usepackage{amsmath,amssymb}
\usepackage{hyperref}
\usepackage{geometry}
\geometry{margin=2.5cm}

% =====================
% TITRE
% =====================
\title{Compression d\textquotesingle{}images JPEG\\[2mm]
	\large Problématique, modèle théorique et compromis qualité/taille}

\author{Khanh-Phuong NGUYEN}
\date{Année universitaire 2025--2026}

\begin{document}
	
	\maketitle
	\tableofcontents
	\clearpage
	
	% ============================================================
	\chapter{Introduction : Problématique de la compression d\textquotesingle{}images}
	% ============================================================
	
	Les images numériques occupent une place centrale dans les échanges actuels
	(photographie, vidéo, réseaux sociaux, archives médicales, etc.). Une image en
	niveaux de gris de taille $N\times M$ codée sur 8 bits par pixel nécessite
	\mbox{$N\times M$~octets} de stockage. Pour des résolutions élevées, le volume
	devient vite incompatible avec les contraintes de bande passante et de mémoire.
	
	Le problème général posé est le suivant~:
	\begin{itemize}
		\item \textbf{Entrée}~: une image discrète $I(x,y)$ définie sur une grille
		$0\le x < N$, $0\le y < M$, avec $I(x,y)\in\{0,\dots,255\}$.
		\item \textbf{Objectif}~: construire un fichier compressé $C$ tel que
		\[
		|C| \ll |I| \quad \text{tout en préservant la qualité visuelle}.
		\]
	\end{itemize}
	
	JPEG choisit délibérément une \emph{compression avec perte}~: l\textquotesingle{}image
	reconstruite $\hat I$ diffère de l\textquotesingle{}originale, mais l\textquotesingle{}erreur est principalement
	portée sur des composantes peu sensibles pour l\textquotesingle{}œil humain (hautes
	fréquences, détails chromatiques fins). Le but de ce rapport est de présenter
	le cadre théorique de cette approche, les outils mathématiques utilisés et le
	compromis obtenu entre qualité et taux de compression.
	
	\section*{Remarques personnelles}
	
	Travailler sur JPEG m\textquotesingle{}a permis de relier des notions assez abstraites
	(transformées, quantification, entropie) à un problème très concret~: stocker
	et transmettre des images dans un contexte de ressources limitées. J\textquotesingle{}ai été
	particulièrement surpris qu\textquotesingle{}une transformée relativement simple comme la DCT,
	associée à une bonne stratégie de quantification, permette d\textquotesingle{}obtenir des
	fichiers beaucoup plus petits sans dégradation flagrante à première vue. Cela
	montre à quel point la perception humaine joue un rôle central dans la
	conception de ces algorithmes.
	
	% ============================================================
	\chapter{Cadre théorique et outils mathématiques}
	% ============================================================
	
	\section{Modèle de l\textquotesingle{}image}
	
	On modélise l\textquotesingle{}image par une matrice de niveaux de gris
	\[
	I = [I(x,y)]_{0\le x<N,\, 0\le y<M}, \qquad I(x,y)\in\{0,\dots,255\}.
	\]
	Les pixels voisins sont fortement corrélés~: cette redondance spatiale permet une
	représentation plus compacte après transformation.
	
	Dans le cas couleur, la norme JPEG recommande de travailler dans l\textquotesingle{}espace
	YCbCr. La composante de luminance $Y$ concentre l\textquotesingle{}information perçue, tandis
	que les chrominances $Cb$ et $Cr$ peuvent être sous-échantillonnées sans perte
	visuelle notable.
	
	\section{Transformée en cosinus discrète (DCT) 2D}
	
	L\textquotesingle{}outil central de JPEG est la DCT appliquée bloc par bloc $(8\times8)$. Chaque
	bloc est recentré autour de zéro~:
	\[
	f(x,y) = I(x,y) - 128, \qquad 0\le x,y<8.
	\]
	
	La DCT 2D est définie pour $u,v\in\{0,\dots,7\}$ par
	\[
	F(u,v) = \frac{1}{4}C(u)C(v)
	\sum_{x=0}^{7}\sum_{y=0}^{7}
	f(x,y)\cos\!\left(\frac{(2x+1)u\pi}{16}\right)
	\cos\!\left(\frac{(2y+1)v\pi}{16}\right),
	\]
	avec
	\[
	C(k)=
	\begin{cases}
		\frac{1}{\sqrt{2}} & \text{si } k=0,\\
		1 & \text{sinon}.
	\end{cases}
	\]
	
	Le coefficient $(0,0)$ (DC) représente la moyenne du bloc. Les autres (AC)
	décrivent des variations spatiales plus ou moins rapides.
	
	\subsection*{Propriété de concentration d\textquotesingle{}énergie}
	
	Pour les images naturelles, la majorité de l\textquotesingle{}énergie est contenue dans les
	basses fréquences : $|F(u,v)|$ décroît en moyenne lorsque $(u,v)$ augmente.
	Cela justifie la forte quantification des hautes fréquences.
	
	\section{Quantification}
	
	Chaque coefficient est quantifié grâce à une table $Q(u,v)$ :
	\[
	F_q(u,v)=\operatorname{round}\!\left(\frac{F(u,v)}{Q_{\text{tab}}(u,v)}\right).
	\]
	
	La norme JPEG définit une table de base $Q_{\text{base}}$ et un facteur de
	qualité $F_q$ via
	\[
	\lambda(F_q)=
	\begin{cases}
		\frac{5000}{F_q}, & F_q<50,\\[1mm]
		200-2F_q, & F_q\ge 50,
	\end{cases}
	\]
	puis
	\[
	Q_{\text{tab}}(u,v)=
	\max\!\Bigl(1,\min\bigl(255,
	\left\lfloor\tfrac{Q_{\text{base}}(u,v)\lambda(F_q)+50}{100}\right\rfloor
	\bigr)\Bigr).
	\]
	
	\section*{Remarques personnelles}
	
	La DCT est assez intuitive lorsqu\textquotesingle{}on la voit comme une décomposition en
	motifs plus ou moins rapides. La quantification demande un changement de
	perspective : accepter volontairement une perte pour gagner en compression.
	
	\section{Réduction de redondance : zig--zag, RLE, Huffman}
	
	Les coefficients quantifiés contiennent de nombreux zéros. JPEG exploite cette
	structure à travers :
	\begin{itemize}
		\item un parcours zig--zag regroupant les zéros en fin de séquence ;
		\item un codage RLE pour les AC ;
		\item un codage différentiel pour les DC ;
		\item un codage de Huffman pour les symboles issus du RLE.
	\end{itemize}
	
	% ============================================================
	\chapter{Chaîne de traitement JPEG}
	% ============================================================
	
	\section{Encodage}
	
	\begin{enumerate}
		\item Prétraitement : conversion YCbCr, découpage en blocs, recentrage.
		\item DCT bloc par bloc.
		\item Quantification des coefficients.
		\item Parcours zig--zag.
		\item RLE et codage différentiel des DC.
		\item Codage de Huffman.
	\end{enumerate}
	
	\section{Décodage}
	
	Inversion des étapes précédentes :
	\begin{enumerate}
		\item décodage Huffman et dé-RLE ;
		\item déquantification ;
		\item IDCT bloc par bloc ;
		\item recentrage inverse et réassemblage.
	\end{enumerate}
	
	% ============================================================
	\chapter{Évaluation de la qualité et du taux de compression}
	% ============================================================
	
	\section{Métriques objectives}
	
	\subsection*{Écart quadratique moyen}
	
	Si $N_{\text{tot}} = N\times M$ est le nombre total de pixels,
	\[
	EQM = \frac{1}{N_{\text{tot}}}\sum_{x,y}\bigl(I(x,y)-\hat I(x,y)\bigr)^2.
	\]
	
	\subsection*{PSNR}
	
	\[
	PSNR = 10\log_{10}\left(\frac{255^2}{EQM}\right)\ \text{dB}.
	\]
	
	\subsection*{Taux de compression}
	
	\[
	T = \frac{\text{taille brute}}{\text{taille JPEG}}.
	\]
	
	\section*{Remarques personnelles}
	
	Les métriques comme le PSNR sont utiles mais ne reflètent pas toujours la
	perception visuelle. Observer les artefacts (blocs, textures perdues) reste
	essentiel.
	
	\section{Influence du facteur de qualité}
	
	Lorsque $F_q$ augmente :
	\begin{itemize}
		\item la quantification s\textquotesingle{}adoucit, le PSNR augmente ;
		\item le taux de compression diminue.
	\end{itemize}
	
	Pour de faibles valeurs de $F_q$, on observe des artefacts visibles.
	
	% ============================================================
	\chapter{Discussion et conclusion}
	% ============================================================
	
	Le schéma JPEG combine :
	\begin{enumerate}
		\item la décorélation via la DCT ;
		\item la réduction perceptuelle via la quantification ;
		\item la compression entropique via Huffman.
	\end{enumerate}
	
	Cette architecture reste très utilisée malgré l’apparition de standards plus
	récents comme JPEG~2000, HEIC ou AVIF.
	
	\begin{thebibliography}{9}
		
		\bibitem{itut81}
		ITU-T Recommendation T.81, \emph{Digital Compression and Coding of
			Continuous-Tone Still Images (JPEG)}, 1992.
		
		\bibitem{jpegNotes}
		Notes de cours ``JPEG 25--26'', support pédagogique sur la chaîne JPEG,
		Université de Poitiers, 2025.
		
	\end{thebibliography}
	
\end{document}
